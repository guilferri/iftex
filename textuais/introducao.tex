\chapter[Introdução]{Introdução}

A análise de fadiga, em si, pode ter diferentes finalidades. É possível que seja buscado quantificar de formar objetiva, como fim por si só, o cansaço de uma pessoa; ou então como meio de valorar o cansaço visual de um usuário de BCI ao visualizar constantemente os estímulos, o que pode contribuir para a melhoria desses dispositivos.

Comumente a fadiga é avaliada através dos ritmos cerebrais fundamentais, compostos pelas bandas de frequência: 0.5-3.5 Hz ($\delta$), 4-7.5 Hz ($\theta$), 8-10 Hz ($\alpha_1$), 10.5-13 Hz ($alpha_2$) e 14-30 Hz ($\beta$) \cite{Craig2012}. Usualmente os valores assumidos